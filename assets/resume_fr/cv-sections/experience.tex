%----------------------------------------------------------------------------------------
%	SECTION TITLE
%----------------------------------------------------------------------------------------

%\includegraphics[scale=2]{emoji_images/hires/1F4BC.pdf}~~~~
\cvsection{Expérience professionnelle}

%----------------------------------------------------------------------------------------
%	SECTION CONTENT
%----------------------------------------------------------------------------------------

\begin{cventries}


%------------------------------------------------

\cventry
{Data Scientist\textbf{*}: Recherches en Neuroscience}% Job title
{SIMEXP lab, CRIUGM / University of Montreal} % Organization
{\hspace{-5mm}Télétravail / Montréal (QC), CANADA} % Location
{Nov. 2018 - Mai 2022} % Date(s)
{ % Description(s) of tasks/responsibilities
	\begin{cvitems}
		\item {Outils logiciels pour la neuroimagerie. \url{https://github.com/SIMEXP}}
		\subitem {Pré-traitement IRMf par HPC et contrôle qualité de données BIDS\cite{lussier2022standardized} (\texttt{fMRIPrep}, \texttt{Datalad}, \texttt{SLURM})}
		\subitem {Soutien long-terme et tests de re-producibilité du logiciel fMRIPrep \cite{chatelain2022testing} (\texttt{Datalad}, \texttt{SLURM})}
		\item {Machine learning. \url{https://github.com/courtois-neuromod}}
		\subitem {Librairie deep learning par graphe pour IRMf: pré-traitement des features, entraînement, évaluation et tests (\texttt{Nilearn}, \texttt{PyTorch})}
		\subitem {Recalage rapide et précis d'images IRMf par quaternions et réseau de convolution profond\cite{tetrel2021fast} (\texttt{TensorFlow})}
		\subitem {Benchmarking de l'entraînement distribuée pour classifer les états cérébraux  \cite{zhang2021functional,zhang2019benchmarking} (collab avec Intel) et annotation audio (\texttt{PyTorch}, \texttt{SLURM})}
		\item {Plateforme de données pour la recherche. \url{https://github.com/neurolibre}}
		\subitem {Admin NeuroLibre \cite{karakuzu2022neurolibre}: Cluster de calcul et API backend pour la publication des soumissions (\texttt{Openstack}, \texttt{Kubernetes/Binderhub})}
		\subitem {Organisation des données, documentation et maintenance de l'infrastructure du laboratoire (\texttt{bash}).}
		\item {Contributions pour logiciels libres (\texttt{TensorFlow}, \texttt{Nilearn}, \texttt{Binderhub}), présentations orales et formateur pour "hackatons" (MAIN, OHBM)}
	\end{cvitems}
}

%------------------------------------------------

\cventry
{Ingénieur en vision par ordinateur\textbf{*}: Solutions 3D pour la dentisterie digitale.}% Job title
{Straumann Group, Digital Business Unit} % Organization
{\hspace{-5mm}Montréal (QC), CANADA} % Location
{Déc. 2016 - Oct. 2018} % Date(s)
{ % Description(s) of tasks/responsibilities
\begin{cvitems}
\item {Algorithmes de reconstruction 3D}
\subitem {État de l'art des méthodes de stéréoscopie par décalage de phase (\texttt{C++}, \texttt{Ceres}, \texttt{Eigen})}
\subitem {Calibration optique et spatiale, correction de distorsion (\texttt{OpenCV}, \texttt{Ceres}, \texttt{NumPy})}
\subitem {Rapports de métrologie et documentation technique}
\subitem {Conception d'un scanner virtuel pour des expérimentations et validation du matériel(\texttt{Blender})}
\item {Participation à des conférences (CVPR 2018, Agile Tour 2017), porte-ouvertes pour recrutement de stagiaires (Concordia, Polytechnique, McGill)}
\end{cvitems}
}

%------------------------------------------------  

\cventry
{Assistant chercheur\textbf{*} : Estimation par graphe de la trajectoire sonde pour l'échographie 3D main libre sans capteurs.}% Job title
{LATIS, ÉTS Montréal} % Organization
{\hspace{-5mm}Montréal (QC), CANADA} % Location
{Jan. 2015 - Nov. 2016} % Date(s)
{ % Description(s) of tasks/responsibilities
\begin{cvitems}
\item {Calibration de capteurs optiques et electromagnetiques pour l'échographie 3D main libre (\texttt{C++}, \texttt{3D slicer/PLUS}, \texttt{Make})}
\item {Mémoire de maîtrise \cite{tetrel2016estimation}: Reconstruction d'images échographiques sans capteurs de position}
\subitem {Recalage d'images échographiques en utilisant la décorrélation du speckle (\texttt{C}, \texttt{Make})}
\subitem {Estimation de la trajectoire par graphe en utilisant des processus gaussiens et l'algèbre de Lie \cite{tetrel2016learning} (\texttt{Matlab}, \texttt{C++}, \texttt{Boost})}
\item {Participations à des conférences (REPARTI 2016, MICCAI/MLMI 2016)}
\end{cvitems}
}

%------------------------------------------------    

\cventry
{Intern\textbf{*} : Initialisation rapide de pistes cartésiennes en utilisant la bande FM} % Job title
{Thales Group, Thales Air Systems} % Organization
{Limours, FRANCE} % Location
{Fév. 2014 - Août 2014} % Date(s)
{ % Description(s) of tasks/responsibilities
\begin{cvitems}
\item {Initialisation de pistes cartésiennes à partir de mesures bi-statiques, en utilisant des filtres non-linéaires et des méthodes statistiques ( \texttt{MATLAB})}
\item {Validation sur des enregistrements réels (\texttt{MATLAB}, \texttt{C++}, \texttt{Eigen})}
\end{cvitems}
}
\end{cventries}
