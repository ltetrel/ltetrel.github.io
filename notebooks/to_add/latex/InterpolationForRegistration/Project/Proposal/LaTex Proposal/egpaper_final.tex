\documentclass[10pt,twocolumn,letterpaper]{article}

\usepackage{cvpr}
\usepackage{times}
\usepackage{epsfig}
\usepackage{graphicx}
\usepackage{amsmath}
\usepackage{amssymb}
\usepackage[utf8]{inputenc}
\usepackage[T1]{fontenc}
\usepackage{lmodern} % load a font with all the characters

% Include other packages here, before hyperref.

% If you comment hyperref and then uncomment it, you should delete
% egpaper.aux before re-running latex.  (Or just hit 'q' on the first latex
% run, let it finish, and you should be clear).
\usepackage[breaklinks=true,bookmarks=false]{hyperref}

\cvprfinalcopy % *** Uncomment this line for the final submission

\def\cvprPaperID{****} % *** Enter the CVPR Paper ID here
\def\httilde{\mbox{\tt\raisebox{-.5ex}{\symbol{126}}}}

% Pages are numbered in submission mode, and unnumbered in camera-ready
%\ifcvprfinal\pagestyle{empty}\fi
\setcounter{page}{1}
\begin{document}

%%%%%%%%% TITLE
\title{Gaussian Process Interpolation for Uncertainty Estimation in Image Registration}

\author{Loïc TETREL\\
McGill / Ecole de technologie supérieure \\
ECSE-626\\
{\tt\small loic.tetrel@mail.mcgill.ca}
% For a paper whose authors are all at the same institution,
% omit the following lines up until the closing ``}''.
% Additional authors and addresses can be added with ``\and'',
% just like the second author.
% To save space, use either the email address or home page, not both
%\and
%Second Author\\
%Institution2\\
%First line of institution2 address\\
%{\tt\small secondauthor@i2.org}
}

\maketitle
%\thispagestyle{empty}

%%%%%%%%% ABSTRACT

%%%%%%%%% BODY TEXT
\section{Introduction}

Image registration technics are widely used in computer vision. It has been shown that interpolating the image's pixels can help to have a more accurate measure when we want to find the pic of similarity \cite{hill2001medical}. The paper of Wachinger \textit{et al.} \cite{wachinger2014gaussian} focuses on a new way to interpolate the images, leading to a new similarity measure that takes into account the incertainty of the interpolation.

%-------------------------------------------------------------------------
\section{Method}

The method is decomposed into three main steps :
\begin{enumerate}
\item A prediction of the interpolation of the image $J$ is done with a prior gaussian process on the resampled image $J^*$.
\item The optimal registration is generated using a bayesian approach with a multivariate Gaussian Likelihood and the prediction of the gaussian process on $J^*$.
\item Reduction of the computationnal cost for 3D volumes and creation of a new uncertainty estimate (not based on $\mathcal{GP}$) to compare the experiments.
\end{enumerate}

From item 2., they derived a new similarity measure taking into account the uncertainty estimates (thanks to the $\mathcal{GP}$ of item 1.).
Finally, they compare there new similarity with the new interpolation to the gold standard interpolation (NN, linear, spline, cubic).

\section{Experimentation}

I plan to code and run the program using MATLAB.
After reading and understading all the paper, I will download the database of patent from the available database BrainWeb and RIRE.
I will constrain my work in the registration of 2D images. I will compute the boxplots (with a statistical analysis) to evaluate the impact of the interpolation, the new similarity measure and the gold standard.
To extend their work, it can be interesting to invest other kernels for the $\mathcal{GP}$ and the impacts of its parameters. Maybe the use of an EM algorithm or variationnal bayesian on item 2. can be used to improve the accuracy of the similarity measure.

{\small
\bibliographystyle{ieee}
\bibliography{egbib}
}

\end{document}
